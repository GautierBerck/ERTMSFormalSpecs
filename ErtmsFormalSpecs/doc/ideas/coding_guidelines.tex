%%%%%%%%%%%%%%%%%%%%%%%%%%%%%%%%%%%%%%%%%%%%%%%%%%%%%%%%%%%%%%%%%%%%%%%%%%%%%%%
%%
%% USAGE:
%%
%% 1. add your name with \addauthor{<initials>}{<full name>} below
%% 2. make your changes
%% 3. for every rule add a \ruleauthor{<initials>}{<date}> at the end
%% 4. put remarks with \<initials>note{<sometext>},
%%    \<initials>warning{<sometext>} or
%%    \<initials>fatal{<sometext>} (in order of significance}
%% 5. for longer texts use a combination of 4. plus
%%    \<initials>explain{<sometext>}
%% 6. add \vhEntry{<versionnumber>}{<date>}{<initials>}{<sometext>} to the
%%    respective section in the header
%%
%%
%%
%% NOTE 1
%% Adding of glossary entries:
%%
%% Make sure to run
%% "makeindex -s coding_guidelines.ist -t coding_guidelines.glg -o
%% coding_guidelines.gls coding_guidelines.glo"
%% after editing the glossaries list
%%
%%
%% NOTE 2
%% Adding of bibliography entries:
%%
%% Edit efsbibliography.bib, cite the new entry, run latex, run bibtex and run
%% latex twice again.
%%
%%
%% NOTE 3
%% Release version:
%%
%% Remove the draft option from the \documentclass-line of this file. This will
%% cause the ruleauthors, the draftwatermarkmark and all the fixme notes to
%% dissappear. Make sure to resolve all fatal-fixmes beforehand.
%%
%%%%%%%%%%%%%%%%%%%%%%%%%%%%%%%%%%%%%%%%%%%%%%%%%%%%%%%%%%%%%%%%%%%%%%%%%%%%%%%


\documentclass[draft]{efsguide}

%% BEGIN title
\renewcommand{\docTitle}{ERTMS Formal Specs Workbench}
\renewcommand{\docSubTitle}{Coding Guidelines}
%% END title

%% BEGIN abbreviations for author names
\addauthor{MD}{Moritz Dorka}
% @ALL: extend as needed
%% END abbreviations for author names

%% BEGIN GLOSSARY
\makeglossaries

\newacronym{efs}{EFS}{ERTMS Formal Specs Workbench}
\newacronym{dsl}{DSL}{Domain Specific Language}
% @ALL: add your entries as necessary. Then follow the instructions from Note 1 above.
%% END GLOSSARY



%%%%%%%%%%%%%%%%%%%%%%%%%%%%%%%%%%%%%%%%%%%%%%%%%%%%%%%%%%%%%%%%%%%%%%%%%%%%%%%
%%
%% BEGIN OF THE ACTUAL DOCUMENT
%%
%%%%%%%%%%%%%%%%%%%%%%%%%%%%%%%%%%%%%%%%%%%%%%%%%%%%%%%%%%%%%%%%%%%%%%%%%%%%%%%

\begin{document}
\title{\docTitle}
\subtitle{\docSubTitle, Version: \vhCurrentVersion}
\date{\vhCurrentDate}
\author{\vhListAllAuthorsLongWithAbbrev}
\maketitle

%% BEGIN VERSIONING
\begin{versionhistory}
\vhEntry{0.1}{31/10/2013}{MD}{initial draft}
\vhEntry{0.2}{12/11/2013}{MD}{add "How to use this file", "Comments", "Glossary", "Bibliography"}
\vhEntry{0.3}{14/12/2013}{MD}{add "Committer checklist"}
% @ALL: add your revision information here
\end{versionhistory}
%% END VERSIONING



\newpage
\tableofcontents
\newpage

\addchap{How to read this file}


This document lists rules to be followed when coding in the \gls{dsl} of the \gls{efs} tool. It is targeted at both model developers and test-case designers. \ldots\\
It will not cover the technical implementation of the underlying \gls{dsl} or give general instructions on how to use the \gls{efs}-tool. For this refer to the Technical design \cite{efstechdesign} and the user guide \cite{efsuserguide}, respectively.\\

In the appendix \ref{appendix:committer_checklist} a short checklist may be found which describes how to correctly prepare a commit for the \href{https://github.com/openETCS/ERTMSFormalSpecs/}{ERTMSFormalSpecs repository on GitHub}.\\

\MDfatal{TODO}


\addsec{Typographical conventions}

\subsection*{Rule texts}
Each rule has a unique identifier composed of the current section and a running number. Words to be used verbatim within the rule texts are printed in \literally{emphasized} letters. \indraftmode{Rule texts conclude with the initials of the rule author and the date of the last change of the rule.}

\subsection*{Code examples}
Code of the \gls{dsl} used for examples within the rules is printed in \texttt{typewriter} font with 
\begin{itemize}
\item language keywords in \texttt{\textbf{bold}},
\item pre-defined functions \texttt{\underline{underlined}},
\item placeholders \texttt{\emph{emphasized}} and
\item string literals in a \texttt{\textsf{sans-serif font}}.
\end{itemize}

For a documentation of the keywords and the pre-defined functions see \cite{efstechdesign}.



\chapter{Domain specific language}

\section{General}
\begin{rules}
\item \MDfatal{TODO} \code{EMPTY} vs. \code{NOT Available()} vs. \code{[]} \ruleauthor{MD}{31/10/2013}
\item \code{SUM List USING X} shall be preferred over \code{REDUCE List USING X + RESULT}. \ruleauthor{MD}{31/10/2013}
\item \code{THERE_IS_IN List | Condition} shall be preferred over \code{MAP List | Condition USING X}. \ruleauthor{MD}{31/10/2013}
\end{rules}


\chapter{Model}

\section{General}

\begin{rules}
\item \label{rule:model_general_design_choices} All decisions made by the modeler which are not purely cosmetic (i.e. using a Function instead of a Procedure), but have an influence on the actual behavior of the model shall be documented in the section \literally{Design Choices} of the Specification View and linked to the respective code. \ruleauthor{MD}{12/11/2013}

\end{rules}


\section{Functions}
\subsection{Cases}

\subsubsection{General}
\begin{rules}
\item \label{rule:functions_cases_general_trivialexpressions}\MDwarning{TBD} Cases shall only contain trivial expressions (i.e. all logic shall be defined inside pre-conditions). \ruleauthor{MD}{31/10/2013}
\item \MDwarning{TBD} Complex expressions and pre-conditions shall not be used together. \MDnote{this rule is mutually exclusive to \ref{rule:functions_cases_general_trivialexpressions}} \ruleauthor{MD}{31/10/2013}
\end{rules}

\subsubsection{Pre-conditions}
\begin{rules}
\item Cases with no pre-condition assigned shall always come as the very last case. \ruleauthor{MD}{31/10/2013}
\item Cases with no pre-condition assigned shall always be named \literally{Otherwise}. \ruleauthor{MD}{31/10/2013}
\end{rules}

\chapter{Tests}

\section{Structure}
\begin{rules}
\item \MDwarning{TBD}\MDexplain{Unfortunately this is not possible in all cases. Require justification if rule cannot be obeyed?} A Sub-step shall either contain Actions or Expectations but not both. \ruleauthor{MD}{12/11/2013}
\item The first Sub-step of the first step of a test frame shall only contain the expression \code{Kernel.InitializeTestEnvironment()}. \ruleauthor{MD}{31/10/2013}
\begin{rules}
\item There shall be no Expectation associated with this Sub-step. \ruleauthor{MD}{31/10/2013}
\end{rules}
\end{rules}


\section{Naming conventions}
\label{rules:tests_naming}
\begin{rules}
\item Individual test steps shall be named \literally{Step n - explanation}.\\ \literally{Explanation} is a user-defined text to summarize the actions and expectations of the current step. \ruleauthor{MD}{31/10/2013}
\end{rules}


\chapter{Comments}

\section{General}
\begin{rules}
\item Comments shall be used wherever the inner workings of a code are not directly obvious from the documentation provided by the linked requirements or design choices. \ruleauthor{MD}{12/11/2013}
\item Comments shall always be assigned to the narrowest possible scope to which they apply (i.e. to an individual test-step rather than to the entire test). \ruleauthor{MD}{12/11/2013}
\item Comments shall not be used to justify Design Choices. See rule \ref{rule:model_general_design_choices} instead. \ruleauthor{MD}{12/11/2013}
\item Comments shall always end with \literally{\#$<$github_username$>$, yy/mm/dd}. Where \literally{$<$github_username$>$} is to be replaced with the actual username of the comment-author and \literally{yy/mm/dd} with the current date. \ruleauthor{MD}{12/11/2013}
\end{rules}


\chapter{Git}
\section{General}
\begin{rules}
\item \label{rule:pullrequest} Own edits shall always be committed into a personal fork of the \gls{efs} repository and then transferred by issuing a pull-request. \ruleauthor{MD}{31/10/2013}
\end{rules}

\section{Commits}
\subsection{General}
\begin{rules}
\item \label{rule:smallcommits} numerous small commits dealing with a single issue are preferable over few huge commits possibly dealing with many different issues. \ruleauthor{MD}{31/10/2013}
\end{rules}

\subsection{commit description texts}
\label{rules:committexts}
\begin{rules}
\item The text shall not contain line-breaks (i.e. consist of a single line only). \ruleauthor{MD}{31/10/2013}
\item Commits dealing with the modeling part of the tool shall begin with \literally{EA_MODEL} followed by a reference to the sections of the subset the commit is related to and a descriptive text. \ruleauthor{MD}{31/10/2013}
\item Commits dealing with the testing part of the tool shall begin with \literally{EA_TEST} followed by a reference to the sections of the subset the commit is related to and a descriptive text. \ruleauthor{MD}{31/10/2013}
\end{rules}

\begin{appendices}

\chapter{Committer checklist}
\label{appendix:committer_checklist}
This is a short checklist that everyone willing to push code onto the ERTMSFormalSpec repository should go through before every commit.

\begin{enumerate}

\item \textbf{make small commits}\\
Each commit should only change one logical thing (i.e. one test-case at a time). See \ref{rule:smallcommits} \ruleauthor{MD}{14/12/2013}
\item \textbf{use meaningful titles for new nodes}\\
For tests see \ref{rules:tests_naming}. For the model: be creative! \ruleauthor{MD}{14/12/2013}
\item \textbf{link to requirements}\\
Always link your tests and model-code with the respective requirements. \ruleauthor{MD}{14/12/2013}
\item \textbf{set implementation-flag}\\
If you implemented any requirements make sure to mark their respective nodes in the specification view with \literally{Implementation Status: Implemented}.\\ \literally{Reviewed: True} should also has to be set if you verified the requirement text matches with that of the original word document of subset-026 \cite{subset26}. \ruleauthor{MD}{14/12/2013}
\item \textbf{check model}\\
Please hit \literally{CTRL+R} in \gls{efs} and see if there are any errors introduced by your work. If so, fix them. \ruleauthor{MD}{14/12/2013}
\item \textbf{commit to your own fork}\\
Commit your changes to your own fork of the ERTMSFormalSpec repository and then create a pull request. See \ref{rule:pullrequest} \ruleauthor{MD}{14/12/2013}
\item \textbf{use meaningful commit messages}\\
It is merely impossible to change them afterwards so please obey \ref{rules:committexts}. \ruleauthor{MD}{14/12/2013}	
\end{enumerate}

\end{appendices}



\phantomsection
\addcontentsline{toc}{chapter}{Glossary}
\printglossary[style=listdotted]
\newpage

\phantomsection
\addcontentsline{toc}{chapter}{Bibliography}
\bibliography{efsbibliography}
\bibliographystyle{plain}


\end{document}