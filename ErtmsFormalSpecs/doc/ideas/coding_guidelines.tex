%%%%%%%%%%%%%%%%%%%%%%%%%%%%%%%%%%%%%%%%%%%%%%%%%%%%%%%%%%%%%%%%%%%%%%%%%%%%%%%
%%
%% USAGE:
%%
%% 1. add your name with \addauthor{<initials>}{<full name>} below
%% 2. make your changes
%% 3. for every rule add a \ruleauthor{<initials>}{<date}> at the end
%% 4. put remarks with \<initials>note{<sometext>},
%%    \<initials>warning{<sometext>} or
%%    \<initials>fatal{<sometext>} (in order of significance}
%% 5. for longer texts use a combination of 4. plus
%%    \<initials>explain{<sometext>}
%% 6. add \vhEntry{<versionnumber>}{<date>}{<initials>}{<sometext>} to the
%%    respective section in the header
%%
%%
%%
%% NOTE 1
%% Adding of glossary entries:
%%
%% Make sure to run
%% "makeindex -s coding_guidelines.ist -t coding_guidelines.glg -o
%% coding_guidelines.gls coding_guidelines.glo"
%% after editing the glossaries list
%%
%%
%% NOTE 2
%% Adding of bibliography entries:
%%
%% Edit efsbibliography.bib, cite the new entry, run latex, run bibtex and run
%% latex twice again.
%%
%%
%% NOTE 3
%% Release version:
%%
%% Remove the draft option from the \documentclass-line of this file. This will
%% cause the ruleauthors, the draftwatermarkmark and all the fixme notes to
%% dissappear. Make sure to resolve all fatal-fixmes beforehand.
%%
%%%%%%%%%%%%%%%%%%%%%%%%%%%%%%%%%%%%%%%%%%%%%%%%%%%%%%%%%%%%%%%%%%%%%%%%%%%%%%%



\documentclass[draft, a4paper, oneside]{scrreprt}
\usepackage[american]{babel}
\usepackage{hyperref}
\usepackage{pdfcomment}
\usepackage{enumitem}
\usepackage[nohyphen]{underscore} %allow for simple _ instead of \_
\usepackage[usenames, dvipsnames]{xcolor}
\usepackage{xspace}
\usepackage[tablegrid]{vhistory}
\usepackage[nonumberlist]{glossaries}
\usepackage{fixme}
\usepackage{ifdraft}
\usepackage[lighttt]{lmodern} %add better ttfont including a bold variant
\usepackage{listings}
\ifdraft{\usepackage[firstpage]{draftwatermark}}{}
\usepackage{scrpage2}


%%
%% BEGIN title + general-stuff
%%

\newcommand{\docTitle}{ERTMS Formal Specs Workbench}
\newcommand{\docSubTitle}{Coding Guidelines}

\hypersetup{%
draft = false,  %link in draft mode as well
pdftitle = {\docTitle, \docSubTitle},
pdfkeywords = {\docTitle, Version \vhCurrentVersion from \vhCurrentDate},
pdfauthor = {\vhAllAuthorsSet},
colorlinks = true,
linkcolor = black,
citecolor = black,
filecolor = black,
urlcolor = black
}

\pagestyle{scrheadings}
\chead[]{}
\ifoot[\docTitle, \docSubTitle\ -- Version \vhCurrentVersion]{\docTitle, \docSubTitle\ -- Version \vhCurrentVersion}
\cfoot[]{}
\ofoot[\thepage]{\thepage}

%%
%% END title + general stuff
%%


%%
%%% BEGIN specific commands for this file
%%

\makeatletter
% makes the enumeration item label start with the number
% of the current section/subsection/etc.
\newcommand{\thecurrentlevel}{%
  \ifnum\c@section=0\thechapter\else
  \ifnum\c@subsection=0\thesection\else
  \ifnum\c@subsubsection=0\thesubsection\else
  \thesubsubsection\fi\fi\fi}
\makeatother

\setcounter{secnumdepth}{3}
\setcounter{tocdepth}{3}
\setlength{\parindent}{0em}

\setenumerate[1]{leftmargin=*, label=\thecurrentlevel.\arabic*.}
\setenumerate[2]{label*=\arabic*.}
\setenumerate[3]{label*=\arabic*.}
\let\emph\textsl



\newcommand{\literally}[1]{\textsf{\emph{#1}}}
\newcommand{\indraftmode}[1]{\ifdraft{#1}{}}

\fxsetup{mode=multiuser, theme=color, layout=inline}
\colorlet{fxnote}{Green} 
\colorlet{fxnotefg}{black}
\colorlet{fxnotebg}{LimeGreen} %is not the same as fxnote to keep it readable
\colorlet{fxfatal}{red}
\colorlet{fxfatalbg}{red}
\colorlet{fxfatalfg}{white}
\colorlet{fxwarning}{Plum}
\colorlet{fxwarningbg}{Plum}
\colorlet{fxwarningfg}{White}
% redefine the layout macro:
\makeatletter
\renewcommand*\FXLayoutInline[3]{%
  \@fxdocolon {#3}{%
    \@fxuseface {inline}%
    \colorbox{fx#1bg}{\color {fx#1fg}\ignorespaces \sffamily #2\textsubscript{\emph{#3}}}}   
    }
\makeatother

\setliststyle{AuthorComment} %alter the appearance of the pdfcomments list

\lstdefinelanguage{EFSDSL}
{%
keywords={SUM, AND, APPLY, COUNT, EMPTY, FIRST_IN, FORALL_IN, IN, INITIAL_VALUE, LAST_IN, MAP, NOT, ON, OR, REDUCE, RESULT, THERE_IS_IN, USING, X},
morekeywords=[2]{AddIncrement, Allocate, Available, Before, DecelerationProfile, IntersectAt, Max, Min, MinSurface, Override, RoundToMultiple,  Targets},
morekeywords=[3]{Boolean, Collection, Condition, double, Expression, f, List, ListExpression, ListValue, Statement, StatementList},
morestring=[b]',
sensitive=true}
\lstset{%
language=EFSDSL,
basicstyle=\ttfamily,
keywordstyle=\bfseries,
keywordstyle=[2]\underline,
keywordstyle=[3]\emph,
stringstyle=\sffamily,
showstringspaces=true,
breakatwhitespace=true,
breaklines=true}
\newcommand{\code}[1]{\lstinline$#1$}

\ifdraft{%
\newcommand{\ruleauthor}[2]{\mbox{}\newline\mbox{}\hfill{\footnotesize\textcolor{gray}{\emph{#1, #2}}}\xspace}}
{%
\newcommand{\ruleauthor}[2]{}
}


\newcommand{\addauthor}[2]{%
%setup vhistory
\expandafter\newcommand\csname #1\endcsname{#2}
%setup fixme
\FXRegisterAuthor{#1}{a#1}{#1}
%setup personal PDF Comment
\ifdraft{%
\expandafter\newcommand\csname #1explain\endcsname[1]{\pdfcomment[author=#2, color=yellow, icon=Comment, subject=Comment, open=true]{##1}}}
{%
\expandafter\newcommand\csname #1explain\endcsname[1]{}
}
}

%%
%% END specific commands for this file
%%


%%
%% BEGIN abbreviations for author names
%%

\addauthor{MD}{Moritz Dorka}

% @ALL:
% extend as needed

%%
%% END abbreviations for author names
%%


%%
%% BEGIN GLOSSARY
%%

\makeglossaries

\newacronym{efs}{EFS}{ERTMS Formal Specs Workbench}
\newacronym{dsl}{DSL}{Domain Specific Language}

% @ALL: add your entries as necessary
% Then follow the instructions from Note 1 above.

%%
%% END GLOSSARY
%%



%%%%%%%%%%%%%%%%%%%%%%%%%%%%%%%%%%%%%%%%%%%%%%%%%%%%%%%%%%%%%%%%%%%%%%%%%%%%%%%
%%
%% BEGIN OF THE ACTUAL DOCUMENT
%%
%%%%%%%%%%%%%%%%%%%%%%%%%%%%%%%%%%%%%%%%%%%%%%%%%%%%%%%%%%%%%%%%%%%%%%%%%%%%%%%

\begin{document}
\title{\docTitle}
\subtitle{\docSubTitle, Version: \vhCurrentVersion}
\date{\vhCurrentDate}
\author{\vhListAllAuthorsLongWithAbbrev}
\maketitle


%%
%% BEGIN VERSIONING
%%

\begin{versionhistory}
\vhEntry{0.1}{31/10/2013}{MD}{initial draft}
\vhEntry{0.2}{12/11/2013}{MD}{add "How to use this file", "Comments", "Glossary", "Bibliography"}
% @ALL:
% add your revision information here

\end{versionhistory}

%%
%% END VERSIONING
%%


\newpage
\tableofcontents
\newpage

\addchap{How to read this file}


This document lists rules to be followed when coding in the \gls{dsl} of the \gls{efs} tool. It is targeted at both model developers and test-case designers. \ldots\\
It will not cover the technical implementation of the underlying \gls{dsl} or give general instructions on how to use the \gls{efs}-tool. For this refer to the Technical design \cite{efstechdesign} and the user guide \cite{efsuserguide}, respectively.\\

\MDfatal{TODO}


\addsec{Typographical conventions}

\subsection*{Rule texts}
Each rule has a unique identifier composed of the current section and a running number. Words to be used verbatim within the rule texts are printed in \literally{emphasized} letters. \indraftmode{Rule texts conclude with the initials of the rule author and the date of the last change of the rule.}

\subsection*{Code examples}
Code of the \gls{dsl} used for examples within the rules is printed in \texttt{typewriter} font with 
\begin{itemize}
\item language keywords in \texttt{\textbf{bold}},
\item pre-defined functions \texttt{\underline{underlined}},
\item placeholders \texttt{\emph{emphasized}} and
\item string literals in a \texttt{\textsf{sans-serif font}}.
\end{itemize}

For a documentation of the keywords and the pre-defined functions see \cite{efstechdesign}.



\chapter{Domain specific language}

\section{General}
\begin{enumerate}
\item \MDfatal{TODO} \code{EMPTY} vs. \code{NOT Available()} vs. \code{[]} \ruleauthor{MD}{31/10/2013}
\item \code{SUM List USING X} shall be preferred over \code{REDUCE List USING X + RESULT}. \ruleauthor{MD}{31/10/2013}
\item \code{THERE_IS_IN List | Condition} shall be preferred over \code{MAP List | Condition USING X}. \ruleauthor{MD}{31/10/2013}
\end{enumerate}


\chapter{Model}

\section{General}

\begin{enumerate}
\item \label{rule:model_general_design_choices} All decisions made by the modeler which are not purely cosmetic (i.e. using a Function instead of a Procedure), but have an influence on the actual behavior of the model shall be documented in the section \literally{Design Choices} of the Specification View and linked to the respective code. \ruleauthor{MD}{12/11/2013}

\end{enumerate}


\section{Functions}
\subsection{Cases}

\subsubsection{General}
\begin{enumerate}
\item \label{rule:functions_cases_general_trivialexpressions}\MDwarning{TBD} Cases shall only contain trivial expressions (i.e. all logic shall be defined inside pre-conditions). \ruleauthor{MD}{31/10/2013}
\item \MDwarning{TBD} Complex expressions and pre-conditions shall not be used together. \MDnote{this rule is mutually exclusive to \ref{rule:functions_cases_general_trivialexpressions}} \ruleauthor{MD}{31/10/2013}
\end{enumerate}

\subsubsection{Pre-conditions}
\begin{enumerate}
\item Cases with no pre-condition assigned shall always come as the very last case. \ruleauthor{MD}{31/10/2013}
\item Cases with no pre-condition assigned shall always be named \literally{Otherwise}. \ruleauthor{MD}{31/10/2013}
\end{enumerate}

\chapter{Tests}

\section{Structure}
\begin{enumerate}
\item \MDwarning{TBD}\MDexplain{Unfortunately this is not possible in all cases. Require justification if rule cannot be obeyed?} A Sub-step shall either contain Actions or Expectations but not both. \ruleauthor{MD}{12/11/2013}
\item The first Sub-step of the first step of a test frame shall only contain the expression \code{Kernel.InitializeTestEnvironment()}. \ruleauthor{MD}{31/10/2013}
\begin{enumerate}
\item There shall be no Expectation associated with this Sub-step. \ruleauthor{MD}{31/10/2013}
\end{enumerate}
\end{enumerate}


\section{Naming conventions}
\begin{enumerate}
\item Individual test steps shall be named \literally{Step n - explanation}.\\ \literally{Explanation} is a user-defined text to summarize the actions and expectations of the current step. \ruleauthor{MD}{31/10/2013}
\end{enumerate}


\chapter{Comments}

\section{General}
\begin{enumerate}
\item Comments shall be used wherever the inner workings of a code are not directly obvious from the documentation provided by the linked requirements or design choices. \ruleauthor{MD}{12/11/2013}
\item Comments shall always be assigned to the narrowest possible scope to which they apply (i.e. to an individual test-step rather than to the entire test). \ruleauthor{MD}{12/11/2013}
\item Comments shall not be used to justify Design Choices. See rule \ref{rule:model_general_design_choices} instead. \ruleauthor{MD}{12/11/2013}
\item Comments shall always end with \literally{\#$<$github_username$>$, yy/mm/dd}. Where \literally{$<$github_username$>$} is to be replaced with the actual username of the comment-author and \literally{yy/mm/dd} with the current date. \ruleauthor{MD}{12/11/2013}
\end{enumerate}


\chapter{Git}
\section{General}
\begin{enumerate}
\item Own edits shall always be committed into a personal fork of the \gls{efs} repository and then transferred by issuing a pull-request. \ruleauthor{MD}{31/10/2013}
\end{enumerate}

\section{Commits}
\subsection{General}
\begin{enumerate}
\item numerous small commits dealing with a single issue are preferable over few huge commits possibly dealing with many different issues. \ruleauthor{MD}{31/10/2013}
\end{enumerate}

\subsection{commit description texts}
\begin{enumerate}
\item The text shall not contain line-breaks (i.e. consist of a single line only). \ruleauthor{MD}{31/10/2013}
\item Commits dealing with the modeling part of the tool shall begin with \literally{EA_MODEL} followed by a reference to the sections of the subset the commit is related to and a descriptive text. \ruleauthor{MD}{31/10/2013}
\item Commits dealing with the testing part of the tool shall begin with \literally{EA_TEST} followed by a reference to the sections of the subset the commit is related to and a descriptive text. \ruleauthor{MD}{31/10/2013}
\end{enumerate}



\phantomsection
\addcontentsline{toc}{chapter}{Glossary}
\printglossary[style=listdotted]
\newpage

\phantomsection
\addcontentsline{toc}{chapter}{Bibliography}
\bibliography{efsbibliography}
\bibliographystyle{plain}

\indraftmode{
\newpage
\phantomsection
\addcontentsline{toc}{chapter}{List of Corrections}
\listoffixmes\newpage
\listofpdfcomments[heading=List of PDF comments]
}

\end{document}